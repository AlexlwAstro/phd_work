
****They are applied to the case of systematic errors in measured fluxes arising as functions of colour. By using a standard set of magnitudes and zero points for a given type of star in the relevant filters, the colour correction terms can be determined by linking the standard values, determined through theoretical stellar models applied to sources with well-constrained observed parameters obtained through repeated observations (such as distance, spectral profile, etc.).\\*


****In general, the formula for colours requiring a first order correction, $b$, in the absence of interference from the Earth's atmosphere, is \citep{2000PASA...17..244S}:
\begin{equation}
 X  =  a  +  x  + b * (X-Y)
\label{color_correction_eq}
\end{equation}
where, for a given type of star, $X$ and $X-Y$ represent the values of the star's apparent magnitude in $X$ and colour $X-Y$, respectively, in the standard filter system. $a$ represents the zero-point of filter $X$ and $x$ represents the apparent magnitude measured by the instruments.
