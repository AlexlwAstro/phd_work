\documentclass{mnras}
\setlength\parindent{0pt}

\usepackage[english]{babel}
\usepackage[utf8x]{inputenc}
\usepackage[T1]{fontenc}

\usepackage{graphicx}
\usepackage{braket}
\usepackage{amsmath}

\par
\begin{document}
\section{Introduction}
\subsection{Motivation}
The fundamental parameters of stars, such as their effective temperatures and metallicities, dictate their observed apparent properties, such as their luminosities and spectra. Hence, a full accounting of the effects of these parameters, and any physical stellar processes that impact on them, directly or indirectly, must be sought.

\subsection{Thermohaline mixing}
The first months of the project were dedicated to the study of thermohaline mixing. This effect was proposed by ****Ulrich (1972) and ****Kippenhahn et al. (1982) to explain anomalous chemical abundances at the surface of mature, ****low-mass red giant branch (RGB) stars. Specifically, the anomalies consist of an over-abundance of 12C, 16O and 14N, together with a paucity of 7Li and 1H, in the stellar spectra. Taken together, these particular changes in these particular species indicate an interaction between the RGB star's fusion shell and the surface, i.e. a mixing effect.

\subsection{Differential extinction}


\section{Current state of the field}
\subsection{Thermohaline mixing}

\subsection{Differential extinction}

\section{Results so far}

\section{Discussion}

\section{Further intended work}

\end{document}